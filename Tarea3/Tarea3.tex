\documentclass[letterpaper,10pt]{article}

% Soporte para los acentos.
\usepackage[utf8]{inputenc}
\usepackage[T1]{fontenc}
% Idioma español.
\usepackage[spanish,mexico, es-tabla]{babel}
% Soporte de símbolos adicionales (matemáticas)
\usepackage{multirow}
\usepackage{amsmath}
\usepackage{amssymb}
\usepackage{amsthm}
\usepackage{amsfonts}
\usepackage{latexsym}
\usepackage{enumerate}
\usepackage{ragged2e}
% Soporte para imágenes.
\usepackage{graphicx}
% Soporte para código.
\usepackage{listings}
% Soporte para URL.
\usepackage{hyperref}
\usepackage[all]{xy} %para diagramas conmutativos
% Modificamos los márgenes del documento.
\usepackage[lmargin=2cm,rmargin=2cm,top=2cm,bottom=2cm]{geometry}

\title{Autómatas y Lenguajes Formales \\ Tarea 3}
\author{González Montiel Luis Fernando \\
        Isai Uzziel García Pérez }
\date{\today}

\begin{document}
\maketitle

\begin{enumerate}

    % Ejercicio 1.
    \item En ciertos lenguajes de programación, los comentarios aparecen entre delimitadores tales
como /\# y \#/. Sea C el lenguaje de todos los comentarios delimitados de forma válida. Un miembro
de C debe empezar con /\# y terminar con \#/ sin que haya en medio ningún \#/. Por simplicidad,
consideramos como alfabeto de las cadenas de entrada a $\Sigma$ = $\lbrace$a,b,/,\#$\rbrace$. Dé una expresión regular
que genere a C. Aquí unos ejemplos en el lenguaje de programación C++(\# es reemplazado por *): \\
	
    \textsc{Solución:}
    \\
    1.- /\#(a+b+/+$\#^{+}$(a+b)$)^{*}$ $\#^{*}$ \#/ \\
    2.- /\#/ *(($\#^{*}$ $a^{+}$ $/^{*}$)+($\#^{*}$ $b^{+}$ $/^{*}$)$)^{*}$ $\#^{*}$ \#/ \\
    3.- /\#((a+b+/$)^{*}$ ($\varepsilon$+$\#^{+}$ (a+b$)^{+}$)$)^{*}$ $\#^{*}$\#/
    
 \end{enumerate}



\end{document}
